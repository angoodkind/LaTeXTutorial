%%%%%%%%%%%%%%%%%%%%%%%%%%%%%%%%%%%%%%%%%
% Beamer Presentation
% LaTeX Template
% Version 1.0 (10/11/12)
%
% This template has been downloaded from:
% http://www.LaTeXTemplates.com
%
% License:
% CC BY-NC-SA 3.0 (http://creativecommons.org/licenses/by-nc-sa/3.0/)
%
%%%%%%%%%%%%%%%%%%%%%%%%%%%%%%%%%%%%%%%%%

%----------------------------------------------------------------------------------------
%	PACKAGES AND THEMES
%----------------------------------------------------------------------------------------

\documentclass[xcolor={svgnames},
hyperref={colorlinks,citecolor=DeepPink4,linkcolor=DarkRed,urlcolor=DarkBlue}
]{beamer}
% usually use \usepackage{hyperref} for link coloring, but in this case beamer
% adds hyperref automatically
%% still have all of the same options

\mode<presentation> {
	
	% The Beamer class comes with a number of default slide themes
	% which change the colors and layouts of slides. Below this is a list
	% of all the themes, uncomment each in turn to see what they look like.
	
	% \usetheme{default}
	% \usetheme{AnnArbor}
	%\usetheme{Antibes}
	%\usetheme{Bergen}
	% \usetheme{Berkeley}
	% \usetheme{Berlin}
	% \usetheme{Boadilla}
	% \usetheme{CambridgeUS} % straightforward, grey title bar
	%\usetheme{Copenhagen}
	%\usetheme{Darmstadt}
	%\usetheme{Dresden}
	%\usetheme{Frankfurt}
	%\usetheme{Goettingen}
	%\usetheme{Hannover}
	%\usetheme{Ilmenau}
	%\usetheme{JuanLesPins}
	%\usetheme{Luebeck}
	% \usetheme{Madrid}
	%\usetheme{Malmoe}
	%\usetheme{Marburg}
	%\usetheme{Montpellier}
	\usetheme{PaloAlto} % has side progress bar
	% \usetheme{Pittsburgh}
	% \usetheme{Rochester}
	% \usetheme{Singapore}
	%\usetheme{Szeged}
	% \usetheme{Warsaw}
	
	% As well as themes, the Beamer class has a number of color themes
	% for any slide theme. Uncomment each of these in turn to see how it
	% changes the colors of your current slide theme.
	
	% \usecolortheme{albatross} % dark blue
	% \usecolortheme{beaver} % minimalist grey/white
	% \usecolortheme{beetle}
	\usecolortheme{crane} % yellow/white
	% \usecolortheme{dolphin} % more white/blue
	% \usecolortheme{dove} % very minimal grey/white
	% \usecolortheme{fly}
	% \usecolortheme{lily}
	% \usecolortheme{orchid}
	% \usecolortheme{rose}
	% \usecolortheme{seagull} % grey
	% \usecolortheme{seahorse} % light blue
	% \usecolortheme{whale}
	% \usecolortheme{wolverine} % yellow/orange
	
	%\setbeamertemplate{footline} % To remove the footer line in all slides uncomment this line
	%\setbeamertemplate{footline}[page number] % To replace the footer line in all slides with a simple slide count uncomment this line
	
	%\setbeamertemplate{navigation symbols}{} % To remove the navigation symbols from the bottom of all slides uncomment this line
}

\usepackage{graphicx} % Allows including images
\usepackage{booktabs} % Allows the use of \toprule, \midrule and \bottomrule in tables
% \usepackage{hyperref}

% for denotation function
\newcommand{\sem}[2][M\!,g]{\mbox{ $[\![ #2 ]\!]^{#1}$}}

% for fancy verbatim options
\usepackage{fancyvrb}

% For syntax trees
\usepackage{qtree}

% For glosses
\usepackage{gb4e}

%----------------------------------------------------------------------------------------
%	TITLE PAGE
%----------------------------------------------------------------------------------------
\title[Short title]{The Joys of \LaTeX} % The short title appears at the bottom of every slide, the full title is only on the title page

\author{Adam Goodkind} % Your name
\institute[Northwestern University] % Your institution as it will appear on the bottom of every slide, may be shorthand to save space
{
	Northwestern University \\ % Your institution for the title page
	\medskip
	\textit{a.goodkind@u.northwestern.edu} % Your email address
}
\date{\today} % Date, can be changed to a custom date
%------------------------------------------------

\begin{document}
	
	\begin{frame}
		\titlepage % Print the title page as the first slide
	\end{frame}
	
	\begin{frame}
		\frametitle{Our Path Today} % Table of contents slide, comment this block out to remove it
		\tableofcontents % Throughout your presentation, if you choose to use \section{} and \subsection{} commands, these will automatically be printed on this slide as an overview of your presentation
	\end{frame}
	
	%----------------------------------------------------------------------------------------
	%	PRESENTATION SLIDES
	%----------------------------------------------------------------------------------------
	
	%------------------------------------------------
	\section{Overview of \LaTeX} % All sections/subsections automatically added to Table of Contents
	\subsection{Why \LaTeX?} 
	%------------------------------------------------
	\begin{frame}
		\frametitle{Why \LaTeX}
		Portability of Code
		\begin{itemize}
			\item <1-> Overleaf.com has made real-time / collaboration simple
			\item <2-> Easy to transfer between templates, layouts and operating systems
			\begin{itemize}
				\item \textbf{What You See Is What You Get...Everywhere!}
			\end{itemize}
			\item <3-> All code is just ASCII text
		\end{itemize}
		Logistics
		\begin{itemize}
			\item <4-> No need to worry about typesetting itself
			\item <5-> Content and layout are kept separate
			\item <6-> No need to worry about where to place figures
			\item <7-> Dynamic section, figure, item numbering etc.
			\item <8-> Citations are automatically added to bibliographies
		\end{itemize}
	\end{frame}
	%------------------------------------------------
	
	\subsection{Why Not \LaTeX?}
	%------------------------------------------------
	\begin{frame}
		\frametitle{Why Not \LaTeX}
		Learning Curve
		\begin{itemize}
			\item <2-> \LaTeX\ is a Turing-complete programming language
			\item <3-> For very (very!) short and straightforward writing, may not be worth overhead
		\end{itemize}
		Tables
		\begin{itemize}
			\item <4-> Tables are a bitch
			\item <5-> Merging/splitting cells can be difficult
			\item <6-> BUT as with everything \LaTeX, extensive online support
		\end{itemize}
		Aesthetics
		\begin{itemize}
			\item <7-> Picture placement/sizing can be tricky
			\item <8-> \LaTeX\ has its fair share of quirks
		\end{itemize}
	\end{frame}
	%------------------------------------------------
	
	\subsection{A Few \LaTeX\ Features}
	%------------------------------------------------
	\begin{frame}[fragile]
		\frametitle{A Few \LaTeX\ Features}
		\begin{itemize}
			\item Very easy to change formats (slide template demo)
			\begin{itemize}
				\item Columns / Format
				\item Line Numbers
				\item Journal Template 		 (\url{https://www.overleaf.com/7657746tdydxffysqjg\#/26800114/})
			\end{itemize}
			\item <2-> Dynamic numbering!!
			\item <3-> Equations and symbols (+Blocks +Verbatim)
			
			\begin{block}<3-> {Example of Denotation Function}
				\begin{Verbatim}[fontsize=\small]
				\usepackage{stmaryrd}
				% or 
				\newcommand{\sem}[2][M\!,g]{\mbox{ $[\![ #2 ]\!]^{#1}$}}
				\sem[M',g]{walks}
				\end{Verbatim}
			\end{block}
			becomes
			$\sem[M',g]{walks}$
		\end{itemize}
	\end{frame}
	%------------------------------------------------
	
	%------------------------------------------------
	\begin{frame}[fragile]
		\frametitle{Tables}
		\begin{block}{Tables can be tricky}
			\begin{columns}
				\begin{column}{0.4\textwidth}
					\begin{Verbatim}[fontsize=\small]
					\begin{table}[]
					\caption{My caption}
					\label{my-label}
					\begin{tabular}{ll}
					Name     & Date   \\
					Adam     & 1/1/17 \\
					Bob      & 1/2/16 \\
					Charlie & 1/3/15
					\end{tabular}
					\end{table}
					\end{Verbatim}
				\end{column}
				\begin{column}{0.4\textwidth}
					\begin{table}[]
						\caption{My caption}
						\label{tbl:my-label}
						\begin{tabular}{|l|c|}
							Name     & Date   \\
							Adam     & 1/1/17 \\
							Bob      & 1/2/16 \\
							Charlie & 1/3/15
						\end{tabular}
					\end{table}
				\end{column}
			\end{columns}
		\end{block}
		
		\begin{itemize}
			\item Lots of tables at \url{http://www.tablesgenerator.com/}
			\item Tables can be made from Python and R adataframes
		\end{itemize}
	\end{frame}
	%------------------------------------------------
	
	\section{Bibliography/ Reference Management}
	\subsection{BibTeX and Biber}
	%------------------------------------------------
	\begin{frame}[fragile]
		\frametitle{From PDF to citation}
		Adding an article from Google Scholar is simple
		\begin{itemize}
			\item Copy the BibTeX key
			\item Paste the key into JabRef
			\item Download the PDF and rename it with the citation key
			\item Add the BibTeX key to the BibTeX file
		\end{itemize}
		
		\begin{block}<2-> {Easy Citations}
			Typing the \verb|\cite| command within \LaTeX:\\~
			\verb|This statement requires citation \cite{p1}.| \\
			Automatically produces
			\begin{itemize}
				\item This statement requires citation \cite{p1}.
				\item And adds it to the bibliography
			\end{itemize}
			\footnotesize{
				\begin{thebibliography}{99} % Beamer does not support BibTeX so references must be inserted manually as below
					\bibitem[Smith, 2012]{p1} John Smith (2012)
					\newblock Title of the publication
					\newblock \emph{Journal Name} 12(3), 45 -- 678.
				\end{thebibliography}
			}
		\end{block}
	\end{frame}
	%------------------------------------------------
	
	\subsection{Mendeley, Zotero and Biber}
	%------------------------------------------------
	\begin{frame}
		\frametitle{Mendeley, Zotero and Biber}
		\begin{itemize}
			\item Mendeley, Zotero, EndNote etc. all support BibTeX
			\item Changing reference styles can be done with a single line
			\item Supposed to use \texttt{biber} instead now
			\begin{itemize}
				\item Offers more stability
				\item Supports more character types
				\item Doesn't work with all bibliography packages, such as \texttt{natbib}
			\end{itemize}
		\end{itemize}
	\end{frame}
	%------------------------------------------------
	
	\section{Presentations and Posters}
	\subsection{Beamer/Tikz}
	%------------------------------------------------
	\begin{frame}
		\frametitle{Posters and Presentations}
		\begin{itemize}
			\item Going between presentations and posters is easy
			\item Can use tikz and beamer packages
			\item A slide is a block
			\item Useful if including lots of technical details
			\item Not always useful for more straightforward presentations
			\item But often less headaches than Microsoft
		\end{itemize}
	\end{frame}
	%------------------------------------------------
	
	\section{\LaTeX\ for Linguists}
	
	%------------------------------------------------
	\begin{frame}[fragile]
		\frametitle{\LaTeX\ For Linguists}
		\textbf{\underline{Lots}} of resources for linguists \\
		\verb|\includegraphics[scale=0.2]{LatexForLinguists.png}|
		\includegraphics[scale=0.2]{LatexForLinguists.png}
	\end{frame}
	%------------------------------------------------
	
	\subsection{Trees}
	%------------------------------------------------
	\begin{frame}[fragile]
		\frametitle{Easy Tree Diagrams}
		This code:
		\begin{Verbatim}
		\usepackage{qtree}
		\Tree [.S 
		[.NP LaTeX ] 
		[.VP [.V is ] [.NP fun ] ] ]
		\end{Verbatim}
		Produces:\\
		\Tree [.S [.NP LaTeX ] [.VP [.V is ] [.NP fun ] ] ]
	\end{frame}
	%------------------------------------------------
	
	\subsection{Symbols}
	%------------------------------------------------
	\begin{frame}
		\frametitle{Lots of Pretty Math}
		Sets \\
		\includegraphics[scale=0.6]{sets.png} \\
		Symbols \\
		\includegraphics[scale=0.4]{symbols.png} \\
		Greek Letters \\
		\includegraphics[scale=0.5]{greek.png}
	\end{frame}
	%------------------------------------------------
	
	\subsection{Glosses}
	%------------------------------------------------
	\begin{frame}[fragile]
		\frametitle{Glosses and Numbering Are Super Easy}
		\begin{itemize}
			\item Example numbers can be continued throughout a document, and modified on-the-fly
			\item Aligning glosses is easy
			\begin{Verbatim}[fontsize=\small]
			\usepackage{gb4e}
			\begin{exe}
			\ex
			\gll Stolz ist er auf seine Kinder gewesen. \\
			proud is he of his children been\\
			\trans `He was proud of his children.'
			\end{exe}
			\end{Verbatim}
			\begin{exe}
				\ex
				\gll Stolz ist er auf seine Kinder gewesen. \\
				proud is he of his children been\\
				\trans `He was proud of his children.'
			\end{exe}
		\end{itemize}
	\end{frame}
	%------------------------------------------------
	
	\section{Extensive Online Support}
	
	%------------------------------------------------
	\begin{frame}
		\frametitle{Support is Everywhere!}
		\begin{itemize}
			\item Ask Anything (General Support)
			\begin{itemize}
				\item \url{tex.stackexchange.com}
				\item \url{reddit.com/r/LaTeX}
				\item \url{en.wikibooks.org/wiki/LaTeX}
			\end{itemize}
			\item For Linguists
			\begin{itemize}
				\item \tiny{\url{http://www.essex.ac.uk/linguistics/external/clmt/latex4ling/}}
				\item \tiny{\url{https://en.wikibooks.org/wiki/LaTeX/Linguistics}}
			\end{itemize}
			\item Symbols Galore (Complete Lists)
			\begin{itemize}
				\item \tiny{\url{http://tug.ctan.org/info/symbols/comprehensive/symbols-a4.pdf}}
				\item Draw! \tiny{\url{http://detexify.kirelabs.org/classify.html}}
				\item Pretty \tiny{\url{http://zelmanov.ptep-online.com/ctan/symbols.pdf}}
			\end{itemize}
			\item Table Generator \url{http://www.tablesgenerator.com/}
			\item Task Management with \href{http://mirrors.ibiblio.org/CTAN/macros/latex/contrib/todonotes/todonotes.pdf}{todonotes}
		\end{itemize}
	\end{frame}
	%------------------------------------------------
	
	%------------------------------------------------
	\begin{frame}
		\Huge{\centerline{The End}}
		\vfill
		\begin{center}
			\footnotesize{Slides \& code available at}
			\footnotesize{\url{https://github.com/angoodkind/LaTeXTutorial}}
		\end{center}
	\end{frame}
	
	%------------------------------------------------
	
\end{document}